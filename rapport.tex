%-------------
% PACKAGES %%
%-------------
\documentclass[12pt]{article}
\usepackage[english]{babel}
\usepackage[utf8x]{inputenc}
\usepackage{hyperref}
\usepackage{amsmath}
\usepackage{graphicx}
\usepackage{float}
\usepackage[colorinlistoftodos]{todonotes}
\usepackage[nottoc,numbib]{tocbibind}

\begin{document}

%-------------------------
%	uncomment irrelevant 
%	parts, like logo etc
%-------------------------

\begin{titlepage}

\newcommand{\HRule}{\rule{\linewidth}{0.5mm}} % Defines a new command for the horizontal lines, change thickness here

\begin{center}
%---------------------
%	HEADING SECTIONS
%---------------------
\textsc{\LARGE Uppsala University}\\[1.5cm] % Name of your university/college
\textsc{\Large Computer Networks and Distributed Systems, 10c}\\[0.5cm] % Major heading such as course name
\textsc{\large }\\[0.5cm] % Minor heading such as course title

%------------------
%	TITLE SECTION
%------------------
\HRule \\[0.4cm]
{ \huge \bfseries Project, 1c}\\[0.4cm] % Title of your document
\HRule \\[1.5cm]
 
%--------------------
%	AUTHOR SECTION
%--------------------
% \begin{minipage}{1.4\textwidth}
% \begin{flushleft} 

\large\emph{Author:}\\Ashraf, Pouya\\
					Billman, Linnar\\
					Boström, Carl\\
					Emrén, Krister\\[1.0cm]% Your name

% \end{flushleft}
% \end{minipage}

%---------------------
%	Supervisor
%---------------------
% \begin{minipage}{0.4\textwidth}
% \begin{flushright} \large
% \emph{Supervisor:} \\
% Dr. James \textsc{Smith} % Supervisor's Name
% \end{flushright}
% \end{minipage}\\[2cm]

% If you don't want a supervisor, uncomment the two lines below and remove the section above
%\Large \emph{Author:}\\
%John \textsc{Smith}\\[3cm] % Your name

{\large \today}\\[2cm] % Date

%----------
%	LOGO 
%----------
% \includegraphics[width=2in]{logga.png}\\%[1cm]
\end{center}
\end{titlepage}
\pagebreak
%-------------------------
%	Table of Contents
%-------------------------
\tableofcontents
\pagebreak

\section{summary} % (fold)
\label{sec:summary}
This project had the aim to explore how one may approach the task of creating a system that runs on a distributed system, and how to solve the problems that occur with that task
% section summary (end)

\section{Introduction} % (fold)
\label{sec:introduction}
For this project we chose to implement a distributed buffer of sorts, which in turn implements a distributed server polling system, to ensure that the system will not fail when the designated server crashes/goes down (when the current server crashes, or loses connection with the connected nodes, a different node will take the erver responsibilities.).
% section introduction (end)

\section{The System} % (fold)
\label{sec:the_system}
The system consists of $<$X$>$ main parts, outlined below.

\subsection{Node.java} % (fold)
\label{sub:node_java}



\subsubsection{The Server Switching Mechanism} % (fold)
\label{ssub:the_server_switching_mechanism}

% subsubsection the_server_switching_mechanism (end)
% subsection node_java (end)
% section the_system (end)

%------------------------------------------
%	picture, copy and paste appropriately
%------------------------------------------
% \begin{figure}[H] %	Absolute positioning
% \centering
% \includegraphics[width=0.8\textwidth]{picture.png}
% \caption{\label{fig:picture}Caption}
% \end{figure}

\end{document}
